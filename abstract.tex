\begin{abstract}
本研究では,類似画像検索技術とテキスト検索を用いた自動画像アノテーションについて述べる.
提案手法は,大きく3段階に分けられる.
1段階目では,Web上からクエリで指定した物体の周辺テキスト付き画像を収集する.
2段階目では,収集した画像がクエリで指定した物体を表しているかどうかの分類を,
類似画像検索処理で行う.
3段階目では選別された画像の周辺テキストを適切な手がかり表現を使ってさらに選別する.
周辺テキストとして日本語が伴った画像を対象とする.
提案手法は
クエリ対象として,犬や猫といった
広いカテゴリの場合よりも
柴犬やシャム猫といった狭いカテゴリを対象とした場合に高い精度を示すことを示した.
%この際,テキスト検索の結果を加味した分類結果としなかった場合の分類結果を比較することで,
%自動画像アノテーションにおける日本語テキストの有用性を示した.
\end{abstract}

% \chapter{概要}
% 本稿では,類似画像検索技術とテキスト検索を用いた自動画像アノテーションについて述べる.
% 提案手法は,大きく二段階に分けられる.
% 一段階目では,Web上からクエリで指定した物体の周辺テキスト付き画像を収集する.
% 二段階目では,収集した画像がクエリで指定した物体を表しているかどうかの分類を,
% 画像特徴量を用いた類似画像検索と周辺テキストを用いたテキスト検索の二手法を用いて行う.
% この際,テキスト検索の結果を加味した分類結果としなかった場合の分類結果を比較することで,
% 自動画像アノテーションにおける日本語テキストの有用性を示した.