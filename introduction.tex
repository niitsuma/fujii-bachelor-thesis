\chapter{はじめに}
近年,多くの画像がWeb 上にアップロードされている.
しかし,計算機はその画像データだけではそれが何を表すものなのか理解するのは困難である.
そのため画像を検索することも非常に困難である.
そこで画像検索のために一般物体認識と呼ばれる技術が求められている\cite{yanai}.

一般物体認識とは,計算機に実世界の画像中の物体やシーンを認識させ,
画像中の位置や物体名称などを一般的な表現で記述する技術である.
画像認識の研究において最も困難な課題の一つとされている.
画像アノテーションはこの一般物体認識の要素課題の1つである.
画像アノテーションは画像が表す内容に対応するメタデータを付与する技術であり,近年活発に研究がなされている\cite{jeon,watanabe}.

画像アノテーションの先駆けとなった研究として森ら\cite{mori}は,百科事典中の画像と説明文から,
画像の部分的な領域と単語の対応を学習することで,
未知の画像から関連する単語を出力する手法を2001年に提案した.
しかし,当時の環境では膨大な認識対象の物体に対して十分なデータ量を用意出来なかったため,認識精度は限定的なものであった.

しかし今日ではインターネットの発展を背景とした,Web上の画像を用いた画像アノテーションの研究が発展している.
例えばImageNet\cite{imagenet}は,WordNetのオントロジーを利用して,その単語の表す物体の画像を人手で収集したデータベースであり,
2012年2月の時点で21,841 の概念,14,197,122 の画像が利用可能である.
ImageNetは人手で画像を分類しているため,非常に誤分類が少なく,機械学習を用いた画像分類の実験によく用いられている. 

これに対して,非常に画素数の少なく画質も悪い画像を大量に収集し,
それらを直接使うことによって画像アノテーション認識を行う事例ベースの手法が提案されている.
例えば,Torralba\cite{torralba}らは,Web検索エンジンを用いて75,062カテゴリ,
約8,000 万枚の画像を収集したTinyImagesと呼ばれるデータベースを用い,
単純な画像特徴量による{\it k}近傍探索を行うことにより,画像アノテーションが可能であることを示した.

本稿では,類似画像検索とテキスト検索を用いた画像アノテーションを用る.
提案手法として,まずWebから物体名をクエリに指定し検索を行い,得られた画像と周辺テキストを収集し,
収集した画像がクエリで指定した物体を表しているかどうかの分類を画像特徴量を用いて行う.
この時,周辺テキストを用いたテキスト検索による上記の分類も行い,
その結果を利用した場合の分類結果としなかった場合の分類結果を比較することで,
自動画像アノテーションにおける日本語テキストの有用性を示す.

本稿は以下の構成をとる.まず2節で自動画像アノテーションの問題定義と説明を行う.
3節で画像アノテーションの関連研究を紹介し,4節で提案手法について説明する.
5節では提案手法を用いた実験の結果および評価について述べる.最後に6節でまとめる.