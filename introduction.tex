\chapter{はじめに}

近年,非常に多くの画像がWeb上にアップロードされている.
Web上の画像は多くの場合は周辺テキストを伴っている.
これら非常に多くの周辺テキストを伴った画像を利用しようする試みの一つとして画像アノテーションがある.
画像アノテーションは画像が表す内容に対応するメタデータを画像に付与する技術であり,
近年活発に研究が行われている\cite{jeon,watanabe}.

画像アノテーションは一般物体認識の要素課題の1つでもある.
% 計算機で検索できるようにしようとする試みが行われている.
% しかし,計算機が多様な画像データの内容を理解するのは困難である.
% そこで計算機が自動で画像を理解するための一般物体認識と呼ばれる技術が求められている\cite{yanai}.
一般物体認識とは,制約のない画像中の物体を検出,認識して
その認識対象の一般的な名称を出力する技術である.
例えば「自転車」が中央に表示されている画像を入力すると「自転車」というテキストが出力がされるようなシステムが一般物体認識の例といえる.
一般物体認識は画像認識の研究において最も困難な課題の一つとされている\cite{yanai}.

%
本研究では,このような画像認識の問題の1つである,
画像認識,画像検出のための学習データを自動収集する課題に注目する.

画像アノテーションの先駆けとなった研究として森ら\cite{mori}は,百科事典中の画像と説明文から,
画像の部分的な領域と単語の対応を学習することで,
未知の画像から関連する単語を出力する手法を2001年に提案した.
しかし,当時の環境では膨大な認識対象の物体に対して十分なデータ量を用意出来なかったため,認識精度は限定的なものであった.
今日ではインターネット上には膨大な画像があり,
Web上の画像を用いた画像アノテーションの研究が発展している.
そのような研究の中で特に有用な研究としてImageNet\cite{imagenet}がある.
ImageNet\cite{imagenet}は,WordNetのオントロジーを利用して,その単語の表す物体の画像を人手で収集したデータセットであり,
2012年2月の時点で21,841個の概念,14,197,122枚の画像が利用可能である.
ImageNetは人手で画像を分類しているため,誤分類が少なく,機械学習を用いた画像分類の実験によく用いられている. 

これに対して,画素数が少なく画質も悪い画像を大量に収集し,
それらを直接使うことによって画像アノテーションを行う事例ベースの手法が提案されている.
例えば,Torralba\cite{torralba}らは,Web検索エンジンを用いて75,062カテゴリ,
約8,000 万枚の画像を収集したTinyImagesと呼ばれるデータベースを用い,
単純な画像特徴量による{\it k}近傍探索を行うことにより,画像アノテーションが可能であることを示した.
%
%画像特徴量とは,画像の持つ特徴の大きさのことである.
%画像アノテーションは,アノテーションを施す画像の特徴を数値として表し,
%他の画像の特徴量と比べることによって実装されることが多い.
%定義は要らない
画像アノテーションによく用いられている特徴量としては,
類似画像検索で良く用いられる色ヒストグラムの他に,
%画像中の物体と背景との境界から物体の形状を特徴量とする
エッジ特徴量や,
%画像の輝度の傾きや強さを特徴量とする
HOG特徴量\cite{dalal}などがある.
%http://www26.atwiki.jp/hirokatsukataoka/pages/19.html

本研究では,類似画像検索とテキスト検索を組み合わせた画像アノテーション手法を提案する.
%提案手法として,
まずTwitterで物体名をクエリに指定し検索を行い,得られた画像と周辺テキストを収集し,
収集した画像がクエリで指定した物体を表しているかどうかの分類を画像特徴量により行う.
この時,周辺テキストを用いたテキスト処理による分類も行い,
その結果を利用した場合の分類結果と利用しなかった場合の分類結果を比較することで,
自動画像アノテーションにおける日本語テキストの有用性を示す.

本稿は以下の構成をとる.
\ref{sec:related}章で画像アノテーションの関連研究を紹介し,
%\ref{sec:method}節で自動画像アノテーションの問題定義と説明を行う.
\ref{sec:way}章で提案手法について説明する.
%\ref{sec:textSearch}
%\ref{sec:way}
\ref{sec:experiment}章では提案手法を用いた実験の結果および評価について述べる.
\ref{sec:examination}章で考察を行い,最後に\ref{sec:format}章でまとめる.