\chapter{まとめ}
\label{sec:format}
本稿では,類似画像検索とテキスト検索を用いた画像アノテーション手法を提案し,
画像アノテーションにおける日本語テキストの有用性を示した.
本手法は,Twitterから画像付きツイートを収集し,
画像特徴量からの類似画像検索と手がかり表現を用いたテキスト検索を併用し,
収集された画像がクエリに指定した物体を表しているかどうかを判別するものである.
提案する自動画像アノテーション手法によって収集された画像は,
画像認識アルゴリズムの学習画像として利用可能となるようなものを選別するようにした.
本手法は
クエリ対象として
犬や猫といった
広いカテゴリの場合よりも
柴犬やシャム猫といった狭いカテゴリを対象とした場合に高い精度を示すことを示した.
% 本手法で3462枚の画像つきツイートから柴犬の画像を分類した結果,
% 429件の画像が得られ,そのうち誤りは2割程度に抑えられた.